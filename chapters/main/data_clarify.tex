%
%   Data Clarification
%       - Plots...
%   
\clearpage
\section{Data clarification}
\label{section:data_clarify}










\begin{figure}[H]
    \centering
    \includegraphics[width=0.5\textwidth]{./graphics/hist_ldate.pdf}
    \caption{Histogram of launch date}
    \label{fig:hist_ldate}
\end{figure}

\inputcode[firstline=78,lastline=80]{R}{rcode/clarification.rmd}

\textbf{[Figure \ref{fig:hist_ldate}]} The \textit{Launch date} histogram shown that most of Intel's CPUs are launched recently, 
with two notable peaks at 2014 and 2017. This would affect tremendously to the bias of the data since some variables are dependent 
on \textit{Launch date}, making the prediction models (\textbf{Section \ref{section:data_analysis}}) weight the recent CPUs heavily,
instead of old, legacy CPUs which are also neccessary to capture a fair pattern over the time.






\begin{figure}[H]
    \centering
    \begin{subfigure}[b]{0.49\textwidth}
        \includegraphics[width=\textwidth]{./graphics/box_litho.pdf}
        \includegraphics[width=\textwidth]{./graphics/sum_litho.png}
        \caption{Box plots and Summary of Lithography}
        \label{fig:box_litho}
    \end{subfigure}
    \hfill
    \begin{subfigure}[b]{0.49\textwidth}
        \includegraphics[width=\textwidth]{./graphics/scatter_litho.pdf}
        \caption{Plot of lithography over time.}
        \label{fig:scatter_litho}
    \end{subfigure}
    \caption{Lithography plots}
\end{figure}

\inputcode[firstline=85,lastline=89]{R}{rcode/clarification.rmd}

\inputcode[firstline=94,lastline=96]{R}{rcode/clarification.rmd}

\textbf{[Figure \ref{fig:box_litho}]} The box plot of \textit{Lithography} (chip printing technique) demonstrates interesting characteristics:
\begin{itemize}
    \item There was a large variance in the types of Lithography designed in Intel's Desktop processors, a smaller, but significant variance of 
    Mobile and Server are also observed. Embedded is less diverse, however, and also concentrates mostly on small \textit{Lithography} printings.

    \item The mean of \textit{Lithography}, in whatever market, is also approximately the same. That number might represents the most suitable 
    printing technique that is widely used among Intel's CPUs.
\end{itemize}

Overall, the most common \textit{Lithography} is $32 nm$. The best printing technique possible was $14 nm$ and the worst was $250 nm$. That large
number might come from old, legacy processors which were not designed with recent innovations in the Chip industry.

\textbf{[Figure \ref{fig:scatter_litho}]} The scatter plot of \textit{Lithography} with respect to \textit{Launch date} shown that the lithography
is getting smaller over time, and they are categorized into specific time intervals. The fact the \textit{Lithography} spans the distribution over
an interval of time instead of condensed into a specific quarter like \textit{Launch date} make it more powerful than \textit{Launch date}. In our
models, we always use \textit{Lithography} instead of \textit{Launch date}. This fact will be visualized with numbers later to make this argument 
more convincing.







\begin{figure}[H]
    \centering
    \includegraphics[width=0.5\textwidth]{./graphics/hist_bfreq.pdf}
    \caption{Histogram of Base Frequency}
    \label{fig:hist_bfreq}
\end{figure}

\inputcode[firstline=108,lastline=110]{R}{rcode/clarification.rmd}

\textbf{[Figure \ref{fig:hist_bfreq}]} The most significant trend that could be observed in the \textit{Base frequency} histogram is its "normality".
It can be seen that, the shape of \textit{Base frequency} disribution is convincingly a bell, and most \textit{Base frequency} concentrates around
the mean (around $2.3$ GHz). This is expected since high \textit{Base frequency} can influence the amound of heat a CPU produces, and there are always
trade-offs between performance and heat. This relationship (between heat and performance) will be visualized in \textbf{Section \ref{section:data_analysis}}.






\begin{figure}[H]
    \centering
    \includegraphics[width=0.5\textwidth]{./graphics/hist_tdp.pdf}
    \includegraphics[width=0.5\textwidth]{./graphics/sum_tdp.png}
    \caption{Box plots and Summary of Thermal Design Power}
    \label{fig:hist_tdp}
\end{figure}

\inputcode[firstline=115,lastline=119]{R}{rcode/clarification.rmd}

\textbf{[Figure \ref{fig:hist_tdp}]} \textit{Thermal design power} is the of main interest in our topic. We can see that, the distribution of \verb|TDP| varies
wildly, from 0W to 300W. However, the occurences of values $\ge 150$W is rare (we can see the 75\%-quantile is at 85W), indicating that we can treat these values 
as outliners and remove them before carrying further analysis.