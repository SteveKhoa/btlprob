%
%   Data Analysis
%       - ANOVA and Regression and stuff.
%   
\clearpage
\section{Data analysis}
\label{section:data_analysis}









\subsection{Analysis of Variance (ANOVA)}

In this section, we analyze the relationships between variables, namely \verb|bfreq| with respect to \verb|litho| and \verb|ncore|.

There are interesting facts about the above attributes:
\begin{itemize}
    \item \verb|bfreq| is a very good representative for CPU's performance, so we will explore
    how other factors contribute to the performance of the CPU.
    \item \verb|litho| represents well for the distinction of each period, in fact, it is
    a better representation for \verb|ldate|. Refer to \textbf{[Figure \ref{fig:scatter_litho}]} we provided in the last section,
    with bare eyes, we can see `litho` is associated with the launch-date very well, and
    it is decreasing over time. What makes it better than \verb|ldate| is that \verb|litho| spans
    over a period of time, and not fixed to a specific year-quarter. One more advantage
    is that, some records do not have launch dates, but they have lithography instead, so using
    it is better to gain more data. We would like to see the impact of decreasing lithography
    on the performance of the CPU.
    \item \verb|ncore| is the driving factor of modern computing, which helps computer utilizing 
    the power of parallelization. This is considered to be a work-around for the approaching lower limit Lithography of 1.5 nanometers
    \cite{2nm-barrier}. On the other hand, more cores also means more cost 
    for a CPU to be manufactured. Because of that, analyzing the number of cores with respect to its perform is neccessary to decide
    whether it is worthy to invest a lot of money produce cores, and whether increasing no. cores impact negatively to the base frequency.
\end{itemize}

Because of that, we chose to analyze them to get a good insight over our CPU dataset.

Refer to \textbf{Figure \ref{fig:box_bfreq_wrt_ncore_litho}}, we can observe that the means of \textit{Base frequency} 
of each \textit{Lithography} were approximately the same when \verb|ncore = 4|, while they were more fluctuated for \verb|ncore = 2|.
From that, we have the following hypotheses:
\begin{enumerate}
    \item For \verb|ncore = 4|, the \textit{Base frequencies} are the same among all groups of \textit{Lithography}.
    \item 
    \item adkgjakdsg
    \item adgkajdhg
\end{enumerate}

\subsubsection{Kruskal-Wallis One-way ANOVA}

\subsubsection{Rank-based Two-way ANOVA}

\subsubsection{Conclusion}





\subsection{Regression models}