\section{Appendix}
\label{section:appendix}

\subsection{Regular Expression}
\label{section:appendix:regex}
The regular expression is used intensively in programming. A regular expression is a sequence of characters that 
specifies a match pattern in text. Usually such patterns are used by string-searching algorithms for "find" 
or "find and replace" operations on strings, or for input validation.

A few special characters used in Perl's Regular expression and their meaning:
\begin{itemize}
    \item \verb|*| : 0 or More, applied on previous character
    \item \verb|+| : 1 or More, applied on previous character
    \item \verb|^| : Beginning of a String
    \item \verb|$| : End of a String
    \item \verb|[ ]| : Matches Characters in brackets
    \item \verb|[^ ]| : Matches Characters NOT in brackets
    \item \verb|( )| : Group
    \item And many more, depends on the extensions.
\end{itemize}

The parser will parse the string, and once they confront these characters, they will perform specific operations
accordingly. In practice, to ensure the regular expression to work as expected and free of bugs before deployment, tools are usually
used, such as \href{https://regex101.com/}{this tool}.

For further usage of Regular Expression in R, refer to \href{https://stat.ethz.ch/R-manual/R-devel/library/base/html/regex.html}{R Documentation page} for more details.









