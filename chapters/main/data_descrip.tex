%
%   Data description
%       - Importing data
%       - Data preprocessing
%       - Data cleaning
%   
\clearpage
\section{Data description}
\subsection{Importing data}
For the first part of our project, we need to select a suitable dataset for us to analyze, as we are computer science students, we have decided to select a CPU data set, the data of which can be found \href{https://www.kaggle.com/datasets/iliassekkaf/computerparts?select=Intel_CPUs.csv}{here}:
\\
The data give us information about 2283 CPU and 45 of their feature which include:
\\
\begin{itemize}
    \item Product\_Collection: tell us which type of series the core belongs to.
    \item Vertical\_Segment: show what kind of system the CPU was designed for (embedded, mobile, desktop, or sever).
    \item Processor\_Number	: process ID.
    \item Status: show the status of the CPU (announce, launched, end of life, end of support).
    \item Launch\_Date: The date the product was first introduced. 
    \item Lithography: refers to the semiconductor technology used to manufacture an integrated circuit, and is reported in nanometers (nm), indicative of the size of features built on the semiconductor. 
    \item Recommended\_Customer\_Price: recommended customer price. 
    \item nb\_of\_Cores: total number of cores in a proccessor. 
    \item nb\_of\_Threads: total number of thread in a processor. 
    \item Processor\_Base\_Frequency: Describes the rate at which the processor's transistors open and close.
    \item Max\_Turbo\_Frequency: The maximum single core frequency at which the processor is capable of operating using Intel® Turbo Boost Technology. 
    \item Cache: CPU Cache is an area of fast memory located on the processor. 
    \item Bus\_Speed: refers to how much data can move across the bus simultaneously.
    \item TDP(thermal design power): Represents the average power, in watts, the processor dissipates when operating at Base Frequency with all cores. 
    \item Embedded\_Options\_Available: is it allow to be embedded system
    \item Conflict\_Free: Defined by the U.S. Securities and Exchange Commission rules to mean products that do not contain conflict minerals (tin, tantalum, tungsten).
    \item Max\_Memory\_Size: The maximum memory capacity supported by the processor. 
    \item Memory\_Types: Single Channel, Dual Channel, Triple Channel, and Flex Mode.The maximum memory capacity supported by the processor.
    \item Max\_nb\_of\_Memory\_Channels: The number of memory channels refers to the bandwidth operation for real world application.
    \item Max\_Memory\_Bandwidth: The maximum rate at which data can be read from or stored into a semiconductor memory by the processor (in GB/s). 
    \item ECC\_Memory\_Supported: ECC memory is a type of system memory that can detect and correct common kinds of internal data corruption.
    \item Processor\_Graphics: integrated graphics processing unit (GPU) that is built into some of Intel's processors.
    \item Graphics\_Base\_Frequency: The rated/guaranteed graphics render clock frequency in MHz.
    \item Graphics\_Max\_Dynamic\_Frequency: The maximum opportunistic graphics render clock frequency (in MHz) that can be supported using Intel HD Graphics with Frequency feature.
    \item Graphics\_Video\_Max\_Memory: The maximum amount of memory accessible to processor graphics. Processor graphics operates on the same physical memory as the CPU (subject to OS, driver, and other system limitations).
    \item Graphics\_Output: Graphics Output defines the interfaces available to communicate with display devices.
    \item Support\_4k: indicates the product's support of 4K
    \item Max\_Resolution\_HDMI: the maximum resolution supported by the processor via the HDMI interface (24bits per pixel \&amp; 60Hz). System or device display resolution is dependent on multiple system design factors; actual resolution may be lower on your system.
    \item Max\_Resolution\_DP: The maximum resolution supported by the processor via the DP interface (24bits per pixel \&amp; 60Hz). System or device display resolution is dependent on multiple system design factors.
    \item Max\_Resolution\_eDP\_Integrated\_Flat\_Panel	
    \item DirectX\_Support: Indicates support for a specific version of DirectX, a Microsoft collection of APIs for handling multimedia compute tasks.
    \item OpenGL\_Support: Indicates support for OpenGL, a cross-language, multi-platform API for rendering 2D and 3D vector graphics. 
    \item PCI\_Express\_Revision: The PCIe version supported by the processor. 
    \item PCI\_Express\_Configurations\_: The available PCIe lane configurations that can be used to link the PCH PCIe lanes to PCIe devices.
    \item T : The maximum temperature allowed on the chip.
    \item Max\_nb\_of\_PCI\_Express\_Lanes: maximum number of PCI Express Lanes that are supported.
    \item Intel\_Hyper\_Threading\_Technology\_: Delivers two processing threads per physical core. Highly threaded applications can get more work done in parallel, completing tasks sooner.
    \item Intel\_Virtualization\_Technology\_VTx\_: Allows one hardware platform to function as multiple “virtual” platforms. It offers improved manageability by limiting downtime and maintaining productivity by isolating computing activities into separate partitions.
    \item Intel\_64\_: Delivers 64-bit computing on server, workstation, desktop and mobile platforms when combined with supporting software. Intel 64 architecture improves performance by allowing systems to address more than 4 GB of both virtual and physical memory.
    \item Instruction\_Set: Which instrution set the CPU use.
    \item Instruction\_Set\_Extensions :  Instruction set extension
    \item Idle\_States: Used to save power when the processor is idle.
    \item Thermal\_Monitoring\_Technologies: Protects the processor package and the system from thermal failure through several thermal management features.	
    \item Secure\_Key: The CPU is supported with secure key or not.
    \item Execute\_Disable\_Bit: Hardware-based security feature that can reduce exposure to viruses and malicious code attacks.
\end{itemize}

For coding the data, our team use a wide range of package which include 
\begin{itemize}
    \item rio: for basic import of data.
    \item ggplot2: for dealing with plot formats.
    \item zoo: for dealing with the year quarter format
    \item ggpubr: to further customize the plot of ggplot
\end{itemize}
\begin{minted}[style=monokai]{R}
   # pacman::p_load(rio,     # for dealing with basic import export
                ggplot2, # for dealing with plot formats
                zoo,     # for dealing with year quarter formats
                ggpubr)  # customize ggplot 

\end{minted}
\subsection{Data preprocessing}
\subsection{Data cleaning}
