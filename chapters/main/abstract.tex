%
%   ABSTRACT
%
%   Total summary of our report.
%   Should include the following information
%       1.  [background] : Briefly introduce the topic, why the report is 
%           relevant to statistics
%       2.  [objectives] : What do we want , what are the aims?
%       3.  [statistical methods] : 
%       4.  [results]
%       5.  [conclusion]
%   
%   Should be written AFTER finishing the report
\section{Abstract}

The development trend of CPU (Central Processing Unit) has caught great interest among the technician
community in recent years. To improve the performance and tackle the heat-tradeoffs, many solutions
and modifications in CPU design has been taken. In this project, we will analyze the data of CPUs
manufactured by Intel as a case study. This data contains almost every information of a complete, usable
CPU, from technical specifications, to recommend prices and intended end-users. Specifically, Thermal Design Power (TDP)
would be our main focus, since it represents the heat barrier and roughly the power consumption of a CPU generation.
We made use of three regression models: Multiple Linear Regression model, Random Forrest Regression model and Logistic Regression
model to examine the relationships between other attributes on Thermal Design Power. Along the way, Analysis of Variance (ANOVA) and
non-parametric tests such as Kruskal-Wallis H-Test and Dunn's test were also use whenever normality of residuals and homoscedasticity
were not satisfied. The results of experimenting multiple models demonstrated that Multiple Linear Regression model was not as good as 
Random Forrest Regression model in predicting the TDP with respect to other attributes. We also found that, the Thermal Design Power in
recent years was tremendously different from the past, indicating a big innovation is going on. In conclusion, by using statistical methods
learnt in class, we got a deeper insight in how to analyze a data from scratch, and make conclusions about the trends and details that is
not apparent.